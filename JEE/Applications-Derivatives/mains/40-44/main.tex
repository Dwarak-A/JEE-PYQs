\iffalse
  \title{Assignment}
  \author{Abhijeet Kumar - ai24btech11001}
  \section{mains}
\fi
%   \begin{enumerate}
\item The maximum volume \brak{in cu.m} of the right circular cone having slant height 3m is 

\hfill \brak{JEE M 2019-9Jan(M)}
\begin{multicols}{2}
\begin{enumerate}
    \item $6\pi$
    \item $3\sqrt{3}\pi$
    \item $\frac{4}{3}\pi$
    \item $2\sqrt{3}\pi$
\end{enumerate}
\end{multicols}
\item If $q$ denotes the acute angle between the curves,
$y=10-x^2$ and $y=2+x^2$ at a point of their intersection, then is equal to: 
\hfill \brak{JEE M 2019-9Jan(M)}
\begin{multicols}{2}
\begin{enumerate}
    \item $\frac{4}{9}$
    \item $\frac{8}{15}$
    \item $\frac{7}{17}$ 
    \item $\frac{8}{17}$
\end{enumerate}
\end{multicols}
\item If $f\brak{x}$ is a non-zero polynomial of degree four, having local extreme end points at $ x=-1,0,1 $; then the set
\\ $S = \{x R:f(x)=f(0)\}$ contains exactly:
\hfill \brak{JEE M 2019-9April(M)}
\begin{enumerate}
    \item four irrational numbers.
    \item four rational numbers.
    \item two irrational and two rational numbers. 
    \item two irrational and one rational number.
\end{enumerate}
\item If the tangent to the curve, $y=x^{3}+ax+b$ at that point $\brak{1,-5}$ is perpendicular to the line, $-x+y+4=0$ then which of the following points lie on the curve ?
\hfill \brak{JEE M 2019-9April(M)}
\begin{multicols}{2}
\begin{enumerate}
    \item $\brak{-2,1}$
    \item $\brak{-2,-2}$
    \item $\brak{2,-1}$ 
    \item $\brak{2,-2 }$
\end{enumerate}
\end{multicols}
\item Let $S$ be the set of all values of $x$ for which the tangent to the curve, $y=f\brak{x}=x^{3}-x^{2}-2x$ at $\brak{x,y}$ is parallel to the line segment joining the points $\brak{1,f\brak{a}}$ and $\brak{-1,f\brak{-1}}$, then $S$ is equal to:

\hfill \brak{JEE M 2019-9April(M)}
\begin{multicols}{2}
\begin{enumerate}
    \item $\{\frac{1}{3},1\}$
    \item $\{\frac{-1}{3},-1\}$
    \item $\{\frac{1}{3},-1\}$ 
    \item $\{\frac{-1}{3},1\}$
\end{enumerate}
\end{multicols}

% \end{enumerate}
