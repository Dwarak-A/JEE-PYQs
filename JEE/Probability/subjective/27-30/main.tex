\iffalse
  \title{Assignment-2}
  \author{EE24BTECH11043-Murra Rajesh Kumar Reddy}
  \section{subjective}
\fi
	\item $A$ is targeting $B$, $B$ and $C$ are targeting to $A$. Probability of hitting the target by $A, B$ and $C$ are $\frac{2}{3},\frac{1}{2}$ and $\frac{1}{3}$ respectively. If $A$ is hit then find the probability that $B$ hits the target and $C$ does not. \hfill${(2003-2 \text{ Marks})}$
	\item $A$ and $B$ are two independent events. $C$ is event in which exactly one of $A$ or $B$ occurs. Prove that $\pr{C}\ge \pr{A\cup B}\pr{\overline{A}\cap\overline{B}}$ \hfill(2004-2 Marks)
	\item A box contains $12$ red and $6$ white balls.Balls are drawn from the box one at a time without replacement. If in 6 draws there are at least $4$ white balls, find the probability that exactly one white drawn in the next two draws. $\brak{\text{binomial coefficients can be left as such}}$ \hfill{(2004-4 Marks)}
\item A person goes to office either by car, scooter,bus or train the probability of which being $\frac{1}{7},\frac{3}{7},\frac{3}{7},\frac{2}{7},$ and$\frac{1}{7}$ respectively. Probability that he reachs office late, ife takes car,scooter,bus or train is $\frac{2}{9},\frac{1}{9},\frac{4}{9}$ and $\frac{1}{9}$ respectively. Goven that he reached office in time, then what is the probability that he travelled by a car. \hfill(2005-2 Marks)



